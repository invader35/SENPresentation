\documentclass[a4paper, parskip=half]{scrartcl} % Dokuemntklasse
\usepackage[utf8x]{inputenc} % Input Encoding
\usepackage{ucs}
\usepackage[ngerman]{babel} % Sprache für Rechtschreibung
\usepackage{amsmath}
\usepackage{amsfonts}
\usepackage{amssymb}
\usepackage{graphicx}  % für Grafiken
\usepackage{listings} % für Source Code Listing
%\usepackage{color}

\title{Beispiel Projekt} % Dokuemnt Titel 
\author{Matthias Schett}
\date{}

%\definecolor{mygreen}{rgb}{0,0.6,0}
%\definecolor{mygray}{rgb}{0.5,0.5,0.5}
%\definecolor{mymauve}{rgb}{0.58,0,0.82}

\lstset{
	frame=single, showstringspaces=false, numbers=left
	%commentstyle=\color{mygreen},    % comment style
	%keywordstyle=\color{blue},       % keyword style
	%numberstyle=\tiny\color{mygray}, % the style that is used for the line-numbers
	%stringstyle=\color{mymauve},     % string literal style
}

\begin{document} % Ab hier beginnt das Dokument

\maketitle  % Erstellt die Standardtitelseite - kann angepasst werden
\tableofcontents % Erstellt das Inhaltsverzeichnis
\newpage % Springt auf einen neue Seite
\section{Neue Section} % Erstellt eine neue Überschrift 
\subsection{Neue Subsection} % Erstellt eine neue Unterüberschrift

Das ist ein neuer Text
Hier gehts einfach weiter

Hier gibts dann einen neuen Absatz
\[ a = b^2 * \sqrt{\frac{c}{d}} \] % Mathe Formel

Die kann man aber auch in einen Satz schreiben $ \int{\log{(\sin{(25x^2))}} \mathrm{d}x} $ 
Aber es sind auch Differentialgleichungen kein Problem:

\[ \frac{d^2}{dt^2}y(t) + 27 \frac{d}{dt} y(t) - 17 y(t) = 23 \sin{(\omega t)} \]

%Quellcodelisting
\begin{lstlisting}
#include <iostream>

using namespace std;

int main(){
	cout << "Hello World to Latex" << endl;

	cin.get();
	return 0;
}
\end{lstlisting}

Nun folgt noch eine kleine Tabelle:

\begin{table}[htb]
	\centering
    \begin{tabular}{ | l | p{7cm} | }
        \hline
        Membername & Bedeutung \\ \hline \hline
        difference\_type & Resultat einer Substraktion des Iterators mit einem anderen \\ \hline
        value\_type & Welchen Typ beihnaltet der Iterator \\ \hline
        pointer & Pointer Typ auf den der Iterator zeigen kann \\ \hline
        reference & Referenz Typ auf den der Pointer zeigen kann \\ \hline
        iterator\_category & Um welcher Iterator Typ handelt es sich (Output, Forward, Random-Access) \\ \hline
    \end{tabular}
    \caption{Member der Traitsklasse}\label{table:traitsMember}
\end{table}

\end{document}